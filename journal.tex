\documentclass[a4paper,journal]{IEEEtran}
%\documentclass[conference]{IEEEtran}
%for using the therefore symbol
%\usepackage{amssymb}
%end

\usepackage[utf8]{inputenc}
\usepackage{graphicx}
\usepackage{float}
\usepackage{color, colortbl}
\usepackage{xcolor}
\usepackage{array}
\usepackage{multirow}
\usepackage{footnote}
\usepackage{cite}
%The below is used to add notes to tables without disrupting the IEEEtran format
\usepackage{threeparttable}
%Multiple figures in a row
%\usepackage{caption}
%\usepackage{subcaption}

% Disable below if wanting to comply exclusively to conference mode of IEEEtran
% \IEEEoverridecommandlockouts

\begin{document}
%opening
 \title{Prototyping Collision-Free MAC Protocols in Real Hardware}


%A more simple output, useful when involving people from different affiliations
  %\author{
    %  \IEEEauthorblockN{Luis Sanabria-Russo\IEEEauthorrefmark{0}, Jaume Barcelo\IEEEauthorrefmark{0}, Boris Bellalta\IEEEauthorrefmark{0}}\\
      %\IEEEauthorblockA{\IEEEauthorrefmark{0}Universitat Pompeu Fabra, Barcelona, Spain
      %\\\{luis.sanabria, jaume.barcelo, boris.bellalta\}@upf.edu}
  %}

\author{Luis Sanabria-Russo \\
		NeTS Research Group at\\
		Universitat Pompeu Fabra, Barcelona, Spain\\
		\texttt{Luis.Sanabria@upf.edu}}

%This is the style of three columns, as indicated in IEEEtran
% \author{\IEEEauthorblockN{Luis Sanabria-Russo}
%  \IEEEauthorblockA{Department of Information\\
%  and Communications Technologies\\
%  Universitat Pompeu Fabra\\
%  Barcelona, Spain\\
%  Email: luis.sanabria@upf.edu}
%  \and
%  \IEEEauthorblockN{Jaume Barcelo}
%  \IEEEauthorblockA{Department of Information\\
%  and Communications Technologies\\
%  Universitat Pompeu Fabra\\
%  Barcelona, Spain\\
%  Email: cristina.cano@upf.edu} 
%  \and
%  \IEEEauthorblockN{Boris Bellalta}
%  \IEEEauthorblockA{Department of Information\\
%  and Communications Technologies\\
%  Universitat Pompeu Fabra\\
%  Barcelona, Spain\\
%  Email: boris.bellalta@upf.edu}}


\maketitle

\begin{abstract}

\boldmath Collisions are a main cause of throughput degradation in WLANs. The current contention mechanism used in this type of network called Carrier Sense Multiple Access with Collision Avoidance (CSMA/CA) uses a Binary Exponential Backoff (BEB) technique to delay each contender attempt of transmitting, effectively reducing the collision probability. Nevertheless, CSMA/CA relies on a random backoff that while effective and totally distributed, in principle is unable to eliminate collisions; degrading the network throughput as more contenders attempt to share the channel. Carrier Sense Multiple Access with Enhanced Collision Avoindance (CSMA/ECA) is able to create a collision-free schedule in a totally distributed manner by means of picking a deterministic backoff after successful transmissions. CSMA/ECA is able to support many contenders in a collision-free schedule, surpassing the achieved throughput of CSMA/CA and provides short-term throughput fairness among contenders.

This work reviews CSMA/ECA, providing insightful simulation results revealing its advantages over CSMA/CA. It also shows the first real-hardware implementation tests results of CSMA/ECA under saturated and unsaturated conditions.

\end{abstract}

\begin{IEEEkeywords}
OpenFWWF, WMP, MAC, Collision-free, CSMA/ECA.
\end{IEEEkeywords}

\section{Introduction}\label{introduction}
Wireless Local Area Networks (WLANs or IEEE 802.11 networks~\cite{802Standards}) are a popular solution for wireless connectivity, whereas in public places, work environments or at home. This technology works over an unlicensed spectrum in the Industrial, Scientific and Medical (ISM) radio bands (at around $2.4$ or $5$~GHz), which is a main reason for its popularity. 

The Medium Access Control (MAC) scheme used in WLANs is called Distributed Coordination Function (DCF) and is based on Carrier Sense Multiple Access with Collision Avoidance (CSMA/CA) protocol. It has been widely adopted by manufacturers and consumers, making it very cheap to implement and an ubiquitous technology. Nevertheless, ever-growing throughput demands from upper layers have proven to be limited by WLANs' MAC, which by its nature is prone to collisions that degrade the overall performance as more nodes join the network.

The research community has pushed forward many alternatives to the current MAC in WLANs~\cite{bharghavan1994map,wang2004ncr,cali2000dti,lopez-toledo2006aoi,
barcelo2008lba,bellalta2009vtc,HE,CSMA_ECA,L_MAC2,hui2011epp,barcelo2011tcf}, but when a proposal deviates too much from CSMA/CA or time-critical operations are modified, its hardware implementation as part of WLANs MAC often becomes unlikely~\cite{WMP}; the standardization process taking many years without certainty of approval~\cite{perahia2008ieee}. 

%Backwards compatibility is paramount, mostly because the existing user base is very big. Also, a suitable candidate to replace CSMA/CA should increase the number of supported contenders while augmenting the offered throughput.



\section{Related Work}\label{relatedWork}
\section{Simulation Results}\label{simulations}
\section{Prototyping CSMA/ECA}\label{prototype}
\section{CSMA/ECA in real hardware results}\label{prototypeResults}
\section{Conclusions}\label{conclusions}
\section{Acknowledgements}


\bibliographystyle{IEEEtran}
\bibliography{../ref}

\end{document}