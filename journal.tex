\documentclass[a4paper,journal]{IEEEtran}
%\documentclass[conference]{IEEEtran}
%for using the therefore symbol
%\usepackage{amssymb}
%end

\usepackage[utf8]{inputenc}
\usepackage{graphicx}
\usepackage{float}
\usepackage{color, colortbl}
\usepackage{xcolor}
\usepackage{array}
\usepackage{multirow}
\usepackage{footnote}
\usepackage{cite}
%The below is used to add notes to tables without disrupting the IEEEtran format
\usepackage{threeparttable}
%Multiple figures in a row
%\usepackage{caption}
%\usepackage{subcaption}
\usepackage{amsmath}

% Disable below if wanting to comply exclusively to conference mode of IEEEtran
% \IEEEoverridecommandlockouts


\usepackage{lipsum}

\begin{document}
%opening
 \title{Prototyping Collision-Free MAC Protocols in Real Hardware}


%A more simple output, useful when involving people from different affiliations
  %\author{
    %  \IEEEauthorblockN{Luis Sanabria-Russo\IEEEauthorrefmark{0}, Jaume Barcelo\IEEEauthorrefmark{0}, Boris Bellalta\IEEEauthorrefmark{0}}\\
      %\IEEEauthorblockA{\IEEEauthorrefmark{0}Universitat Pompeu Fabra, Barcelona, Spain
      %\\\{luis.sanabria, jaume.barcelo, boris.bellalta\}@upf.edu}
  %}

\author{Luis Sanabria-Russo \\
		NeTS Research Group at\\
		Universitat Pompeu Fabra, Barcelona, Spain\\
		\texttt{Luis.Sanabria@upf.edu}}

%This is the style of three columns, as indicated in IEEEtran
% \author{\IEEEauthorblockN{Luis Sanabria-Russo}
%  \IEEEauthorblockA{Department of Information\\
%  and Communications Technologies\\
%  Universitat Pompeu Fabra\\
%  Barcelona, Spain\\
%  Email: luis.sanabria@upf.edu}
%  \and
%  \IEEEauthorblockN{Jaume Barcelo}
%  \IEEEauthorblockA{Department of Information\\
%  and Communications Technologies\\
%  Universitat Pompeu Fabra\\
%  Barcelona, Spain\\
%  Email: cristina.cano@upf.edu} 
%  \and
%  \IEEEauthorblockN{Boris Bellalta}
%  \IEEEauthorblockA{Department of Information\\
%  and Communications Technologies\\
%  Universitat Pompeu Fabra\\
%  Barcelona, Spain\\
%  Email: boris.bellalta@upf.edu}}


\maketitle

\begin{abstract}

\boldmath Collisions are a main cause of throughput degradation in WLANs. The current contention mechanism used in this type of network called Carrier Sense Multiple Access with Collision Avoidance (CSMA/CA) uses a Binary Exponential Backoff (BEB) technique to delay each contender attempt of transmitting, effectively reducing the collision probability. Nevertheless, CSMA/CA relies on a random backoff that while effective and totally distributed, in principle is unable to eliminate collisions; degrading the network throughput as more contenders attempt to share the channel. Carrier Sense Multiple Access with Enhanced Collision Avoindance (CSMA/ECA) is able to create a collision-free schedule in a totally distributed manner by means of picking a deterministic backoff after successful transmissions. CSMA/ECA is able to support many contenders in a collision-free schedule, surpassing the achieved throughput of CSMA/CA and providing short-term throughput fairness among contenders.

This work describes CSMA/ECA and its mechanisms to achieve a collision-free schedule with many contenders by providing insightful simulation and real implementation results revealing its advantages over CSMA/CA. 
%It also shows the first real-hardware implementation tests results of CSMA/ECA under saturated and unsaturated conditions.

\end{abstract}

\begin{IEEEkeywords}
CSMA/ECA, WLAN, MAC, Collision-free, OpenFWWF.
\end{IEEEkeywords}

\section{Introduction}\label{introduction}
Wireless Local Area Networks (WLANs or IEEE 802.11 networks~\cite{802Standards}) are a popular solution for wireless connectivity, whether in public places, work environments or at home. This technology works over an unlicensed spectrum in the Industrial, Scientific and Medical (ISM) radio bands (at around $2.4$ or $5$~GHz), which is a main reason for its popularity. 

The Medium Access Control (MAC) scheme used in WLANs is called Distributed Coordination Function (DCF) and is based on Carrier Sense Multiple Access with Collision Avoidance (CSMA/CA) protocol. It has been widely adopted by manufacturers and consumers, making it very cheap to implement and an ubiquitous technology. Nevertheless, the ever-growing throughput demands from upper layers have proven to be limited by WLANs' MAC~\cite{perahia2008ieee}, which by its nature is prone to collisions that degrade the overall performance as more nodes join the network.

The research community has pushed forward many alternatives to the current MAC in WLANs~\cite{bharghavan1994map,wang2004ncr,cali2000dti,lopez-toledo2006aoi,
barcelo2008lba,bellalta2009vtc,HE,CSMA_ECA,L_MAC2,hui2011epp,barcelo2011tcf}, but when a proposal deviates too much from CSMA/CA or time-critical operations are modified, its hardware implementation as part of WLANs' MAC often becomes unlikely~\cite{WMP}; the standardization process taking many years without certainty of approval~\cite{perahia2008ieee}. 

%Backwards compatibility is paramount, mostly because the existing user base is very big. Also, a suitable candidate to replace CSMA/CA should increase the number of supported contenders while augmenting the offered throughput.

A CSMA/CA replacement should be able to provide advantages in terms of throughput and handle many contenders without sacrificing short-term throughput fairness. Furthermore, to support the existing user base and ease its implementation on real hardware, the new MAC protocol should be built on top of the current standard, ensuring backwards compatibility and avoiding a drastic deviation from CSMA/CA.

%Furthermore, due to the proliferation of evermore WiFi-capable devices, it is thought that a suitable replacement should 

A suitable candidate, and the one to be tested in this work, is called Carrier Sense Multiple Access with Enhanced Collision Avoidance (CSMA/ECA)~\cite{barcelo2008lba}. It is capable of attaining higher throughput than CSMA/CA by making a simple modification to the contention mechanism. In CSMA/ECA, nodes pick a deterministic backoff after successful transmissions; constructing a collision-free schedule among successful contenders. Further enhancements, like \emph{Hysteresis} and \emph{Fair Share}~\cite{research2standards} allowed CSMA/ECA to support many more contenders in a collision-free schedule without compromising short-term fairness.

Although many studies have been made analyzing the performance of CSMA/ECA~\cite{barcelo2008lba,research2standards,bellalta2009vtc,E2CA_performance}, neither assesses the protocol's backwards compatibility property under different traffic conditions. Furthermore, all the aforementioned studies are based on simulation results, bypassing the influence of realistic testing scenarios over the overall network performance.

This work provides the first performance analysis of CSMA/ECA~\cite{research2standards} under unsaturated conditions. Furthermore, CSMA/ECA is prototyped in real hardware using OpenFWWF~\cite{OpenFWWF} and the impact of CSMA/ECA nodes in a real CSMA/CA network is scrutinized for the first time.

The rest of this work is divided as follows: an overview of different MAC protocols for WLANs is provided in  Section~\ref{relatedWork}. CSMA/ECA, as well as its properties for allocating many contenders in a collision-free schedule are explained in Section~\ref{introProtocol}. Section~\ref{simulations} details the simulation procedure for testing CSMA/ECA under unsaturated network conditions, while Section~\ref{prototype} goes through the prototyping of CSMA/ECA in real hardware. The results for the simulation and prototypes are presented in Section~\ref{prototypeResults}. Conclusions are drawn in Section~\ref{conclusions}.


%{\color{blue}\lipsum}

\section{Related Work}\label{relatedWork}
Big advances in the WLANs PHY~\cite{perahia2008ieee,6191306} push the community towards the development of MAC protocols able to take advantage of a much faster PHY.

\subsection{Zero Collision MAC} 

Zero Collision MAC (ZC MAC)~\cite{ZMAC} achieves a zero collision schedule for WLANs in a totally distributed way. It does so by allowing contenders to reserve one empty slot ($s_{e}$) from a  predefined virtual schedule of $N$-slots length. Backlogged stations pick a slot in the virtual cycle to attempt transmission. If two or more stations pick the same slot in the cycle, their transmissions will eventually collide; forcing the involved contenders to randomly and uniformly select other empty slot from those detected in the previous cycle plus the slot where they collided. When all $M$ stations reserve a different slot, a collision-free schedule is achieved.

ZC MAC is able to outperform CSMA/CA under different scenarios. Nevertheless, given that the length of ZC MAC's virtual cycle has to be predefined without actual knowledge of the real number of contenders in the deployment, the protocol is unable to provide a collision-free schedule when $M>N$. Furthermore, if $N$ is overestimated ($N>>M$), the fixed-width empty slots between each contender successful transmission are no longer negligible and contribute to the degradation of the network performance.

%Important to see how L-MAC leverages this previous information requirement and translates into a clock-drift issue.

\subsection{Learning-MAC}

Learning-MAC~\cite{L_MAC} is another interesting MAC protocol which is able to build a collision-free schedule for many contenders. It does so defining a \emph{learning strength} parameter, $\beta\in(0,1)$. Each contender starts by picking a slot for transmission $s$ of the schedule $n$ of length $C$ at random with uniform probability. After a contender picks slot $s(n)$, its selection in the next schedule ($s(n+1)$) will be conditioned by the result of the current attempt. Equation~\ref{success} and Eq.~\ref{collision} show the probability of selecting the same slot $s(n)$ in cycle $n+1$.

\begin{equation} \label{success}
		\left. \begin{aligned}
			p_{s(n)}(n+1)&=1,\\
			p_{j}(n+1)&=0,
		\end{aligned}
	\right\}
	\qquad \text{\emph{Success}}
\end{equation}
\begin{equation} \label{collision}
	\left. \begin{aligned}
			p_{s(n)}(n+1)&=\beta p_{s(n)}(n),\\
			p_{j}(n+1)&=\beta p_{j}(n)+\frac{1-\beta}{C -1},
		\end{aligned}
	\right\}
	\qquad \text{\emph{Collision}}
\end{equation}
\\
, for all $j\neq s(n),~j\in \{1,\dots ,C\}$. That is, if a station successfully transmitted in $s(n)$, it will pick the same slot on the next schedule with probability one. Otherwise, the selection follows Eq.~\ref{collision}.

The selection of $\beta$ implies a compromise between fairness and convergence speed, which the authors determined $\beta=0.95$ to provide satisfactory results.

L-MAC is able to achieve better levels of throughput than the current MAC with a very fast convergence speed. Nevertheless, the choice of $\beta$ suppose a previous knowledge of the number of empty slots ($C-N$, where $N$ is the number of contenders), which is not easily available to the current MAC or may require a centralized entity~\cite{barcelo2011tcf}.

Further extensions to L-MAC introduced an \emph{Adaptative} schedule length in order to increase the number of supported contenders in a collision-free schedule. This adaptive schedule length ($C_{i}$) is doubled or halved depending on the presence of collisions or many empty slots per schedule, respectively. The effects of reducing the schedule length may provoke a re-convergence phase which can result in short-term fairness issues. Furthermore, L-MAC is unable to achieve a collision-free schedule unless $N\leq C$.

\section{Carrier Sense Multiple Access with Enhanced Collision Avoidance (CSMA/ECA)}\label{introProtocol}
	\subsection{Co-existing with CSMA/CA}
\section{Simulation}\label{simulations}
\section{Prototyping CSMA/ECA}\label{prototype}
\section{Coexistence with CSMA/CA}\label{prototypeResults}
\section{Conclusions}\label{conclusions}
\section{Acknowledgements}


\bibliographystyle{IEEEtran}
\bibliography{../ref}

\end{document}