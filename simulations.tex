\section{Simulation Scenario}\label{simulations}
This section provides the simulation parameters for testing CSMA/ECA$_{\text{Hys+FS}}$ under two different traffic conditions, namely saturated and non-saturated. \textcolor{red}{We also provide details on how channel imperfections are modelled and what are its effects over the transmissions}. Further, the simulation of the clock drift effect, and the coexistence with CSMA/CA are also subjects to be addressed in this section.

	\subsection{Scenario details}
	Results are obtained by running multiple simulations over a modified version of the COST simulator~\cite{COST}, available at~\cite{sim:parameters}. PHY and MAC parameters are detailed in Table~\ref{tab:mac-params}. Some assumptions were made in order to test the performance at the MAC layer:
	
	\begin{itemize}
		\item Unspecified parameters follow the IEEE 802.11n ($2.4$~GHz) standard.
		\item All nodes are in communication range.
		\item No Request-to-Send (RTS) or Clear-to-Send (CTS) messages are used.
		\item Collisions take as much channel time as successful transmissions.
	\end{itemize}
	
	The aforementioned assumptions ensure that the simulation results are just effects of the MAC behaviour.
	%Further, in a collision-free CSMA/ECA$_{\text{Hys+FS}}$ operation there is no need for RTS/CTS mechanisms, given that successful nodes already know their respective transmission slot. \textcolor{red}{Maybe here is where the conversation about hidden nodes should take place?}
	If not mentioned otherwise, results are derived from \textcolor{red}{20 simulations of 100 seconds in length, each one with a different seed}. Figures also show the standard deviation.
	
	\begin{table}
		\centering
		\caption{PHY and MAC parameters for the simulations \textcolor{red}{REVIEW}}
		\label{tab:mac-params}
		\begin{tabular}{|c|c|}
			\hline
			\multicolumn{2}{|c|}{{\bfseries PHY}}\\
			\hline
			{\bfseries Parameter} & {\bfseries Value}\\
			\hline
			PHY rate & 65~Mbps\\
			Empty slot & $9~\mu s$\\
			DIFS & $28~\mu s$\\
			SIFS & $10~\mu s$\\
			\hline
			\multicolumn{2}{|c|}{{\bfseries MAC}}\\
			\hline
			{\bfseries Parameter} & {\bfseries Value}\\
			\hline
			Maximum backoff stage ($m$) & 5\\
			Minium Contention Window ($CW_{\min}$) & 16\\
			Maximum retransmission attempts & 6\\
			Data payload (Bytes) & 1024\\
			MAC queue size (Packets) & 1000\\
			\hline
		\end{tabular}
	\end{table}
	
	\subsection{Saturated and Non-saturated stations}\label{unsaturation}
	A saturated station always has packets in its MAC queue. This is modelled by setting the packet arrival rate to the MAC queue ($\Delta_{\text{PAR}}$) to a value greater than the achievable throughput. To ensure saturation, stations are set to fill their MAC queue at $\Delta_{\text{PAR}}=65$~Mbps, which is purposefully greater than the effective capacity of the channel.
	
	To evaluate the performance under non-saturated conditions, stations need to be able to empty their MAC queues. To do so, the packet arrival rate to the MAC queue is set to $\Delta_{\text{PAR}}=1$~Mbps. These values of $\Delta_{\text{PAR}}$ have proven to produce the desired effects.
	
	\subsection{Performance under clock drift}
	Clock drift is simulated by setting a drift probability, $p_{cd}$. Each station has a probability of $p_{cd}/2$ of miscounting one slot more, and $p_{cd}/2$ of miscounting one slot less. This approach follows the one proposed by Gong et. al in~\cite{slotDrift}.
	
	\textcolor{red}{\subsection{Channel errors}\label{channelErrorsDef}
	Channel errors are modelled by assigning a probability of a packet being corrupted by the channel, $p_e>0$. That is, in a single packet transmission there is probability $p_e$ that the transmission will not be acknowledged. If the transmission is an A-MPDU (like in the case of CSMA/ECA$_{\text{Hys+FS}}$), $p_e$ will affect each packet individually and independently. Therefore, an A-MPDU transmission will be considered a failure if all packets in the aggregation are affected by the channel error probability.}
	
	\textcolor{red}{All results are shown with stickiness equal to one (see Section~\ref{errorEffect}). That is, after a colision, CSMA/ECA$_{\text{Hys+FS}}$ nodes will use a random backofff. Nevertheless, in Section~\ref{resultsSchedRest} curves described as \emph{dynStick} temporarily increase the node's stickiness to two after a successful reduction of the schedule (using a random backoff after two consecutive collisions). This increase is done in order to converge faster to collision-free schedules when in operating with channel errors.}
	
	\subsection{Coexistance with CSMA/CA}\label{coexistence}
	To test the performance of CSMA/CA and CSMA/ECA$_{\text{Hys+FS}}$ stations in the same network, simulations are set with a CSMA/CA node density of 1/4, 1/2 and 3/4 of the total.
	
	\textcolor{red}{\subsection{Applying Schedule Reset} 
	A set of results under saturated conditions and channel errors applying Schedule Reset (see Section~\ref{schedReset}) are provided. Some of the results are generated with a $\gamma=1$, or \emph{aggressive} Schedule Reset (aggr. in the figures). These settings provide the highest throughput under the tested conditions, and also in the real hardware implementations such as the one shown in Section~\ref{EDCA}.
	}
