\section{Related Work}\label{relatedWork}
Time in WLANs is divided into tiny empty slots of fixed length $\sigma_{e}$, collisions, and successful slots of length $\sigma_{c}$ and $\sigma_{s}$, respectively. Collision and successful slots contain collisions or successful transmissions, making them several orders of magnitude larger than empty slots ($\sigma_{e}\ll\min(\sigma_{s},\sigma_{c}))$. One of the effects of collisions is the degradation of the network performance by wasting channel time on collisions slots. 

Recent advances in the WLANs PHY~\cite{perahia2008ieee,6191306} push the research community towards the development of MAC protocols able to take advantage of a much faster PHY. By reducing the time spent in collisions nodes are able to transmit more often, which in turn translates to an increase in the network throughput. Further, the upcoming MAC protocols for WLANs should work without message exchange between contenders, that is, work in a fully decentralised fashion in order to avoid injecting extra control traffic that may reduce the data throughput.

\textcolor{black}{Performing time slot reservation for each transmission is a well known technique for increasing the throughput and mantanining Quality of Service (QoS) in TDMA schemes, like LTE~\cite{canoLTEcoexistence}. Applying the same concept to CSMA networks by modifying DCF's random backoff proceedure provides similar benefits~\cite{HE}. The following are MAC protocols for WLANs, decentralised and capable of attaining greater throughput than CSMA/CA by constructing collision-free schedules using reservation techniques. A survey of collision-free MAC protocols for WLANs is presented in~\cite{L_MAC}. In this paper we only overview those that are similar to CSMA/ECA.}

\subsection{Zero Collision MAC}\label{ZC-MAC}

Zero Collision MAC (ZC-MAC)~\cite{ZMAC} achieves a zero collision schedule for WLANs in a fully decentralised way. It does so by allowing contenders to reserve one empty slot from a predefined virtual schedule of $M$-slots in length. Backlogged stations pick a slot in the virtual cycle to attempt transmission. If two or more stations picked the same slot in the cycle, their transmissions will eventually collide. This forces the involved contenders to randomly and uniformly select other empty slot from those detected empty in the previous cycle plus the slot where they collided. When all $N$ stations reserve a different slot, a collision-free schedule is achieved.

ZC-MAC is able to outperform CSMA/CA under different scenarios. Nevertheless, given that the length of ZC MAC's virtual cycle has to be predefined without actual knowledge of the real number of contenders in the deployment, the protocol is unable to provide a collision-free schedule when $N>M$. Furthermore, if $M$ is overestimated ($M\gg N$), the fixed-width empty slots between each contender's successful transmission are no longer negligible and contribute to the degradation of the network performance. Additionally, ZC-MAC nodes require common knowledge of where the virtual schedule starts/ends. This is not considered in CSMA/CA and constitutes an obstacle towards standardisation.

\subsection{Learning-MAC}

Learning-MAC~\cite{L_MAC} is another MAC protocol able to build a collision-free schedule for many contenders. It does so defining a \emph{learning strength} parameter, $\beta\in(0,1)$. Each contender starts by picking a slot $s$ for transmission of the schedule $n$ of length $C$ at random with uniform probability. After a contender picks slot $s(n)$, its selection in the next schedule, $s(n+1)$, will be conditioned by the result of the current attempt. (\ref{success}) and (\ref{collisions-eq}) extracted from~\cite{L_MAC} show the probability of selecting the same slot $s(n)$ in cycle $n+1$.

\begin{equation} \label{success}
		\left. \begin{aligned}
			p_{s(n)}(n+1)&=1,\\
			p_{j}(n+1)&=0,
		\end{aligned}
	\right\}
	\qquad \text{\emph{Success}}
\end{equation}
\begin{equation} \label{collisions-eq}
	\left. \begin{aligned}
			p_{s(n)}(n+1)&=\beta p_{s(n)}(n),\\
			p_{j}(n+1)&=\beta p_{j}(n)+\frac{1-\beta}{C -1},
		\end{aligned}
	\right\}
	\qquad \text{\emph{Collision}}
\end{equation}
\\
for all $j\neq s(n),~j\in \{1,\dots ,C\}$. That is, if a station successfully transmitted in $s(n)$, it will pick the same slot on the next schedule with probability one. Otherwise, it follows~(\ref{collisions-eq}).

The selection of $\beta$ implies a compromise between fairness and convergence speed, which the authors determined $\beta=0.95$ to provide satisfactory results.

L-MAC is able to achieve higher throughput than CSMA/CA with a very fast convergence speed. Nevertheless, the choice of $\beta$ suppose a previous knowledge of the number of empty slots ($C-N$, where $N$ is the number of contenders), which is not easily available to CSMA/CA or may require a centralised entity~\cite{barcelo2011tcf}.

Further extensions to L-MAC introduced an \emph{Adaptative} schedule length in order to increase the number of supported contenders in a collision-free schedule. This adaptive schedule length is doubled or halved depending on the presence of collisions or many empty slots per schedule, respectively. The effects of reducing the schedule length may provoke a re-convergence phase which can result in short-term fairness issues. Moreover, L-MAC nodes also require common knowledge of the start/end of the schedule.